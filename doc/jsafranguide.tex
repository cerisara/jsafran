\documentclass{article}

\usepackage{url}

\begin{document}

\section{Introduction}
For now, this is just a sketch of a document.

A somehow more complete (though older) guide can be found on the web (\url{http://rapsodis.loria.fr/jsafran/userguide.html}).

\section{Search/Replace function}

The search/replace function manipulates trees, and can be accessed from the menu,
or from the F2 shortcut, or yet the Ctrl-F shortcut (useful especially for Mac).

In the following, we call a {\bf triplet} something in the form:
\begin{verbatim}
f=je,p=*VER.*,SUJ
\end{verbatim}
or with optional parenthesis, such as in:
\begin{verbatim}
(f=je,_,_)
\end{verbatim}

A triplet is composed of 3 fields:
\begin{enumerate}
\item The governed word (on both previous examples, constrained to be of the form ``je'', case-insensitive; use ``F='' to be case-sensitive, ``l='' to match the lemma and ``p='' to match the POS-tag; use ``!='' to match any word except...; use ``\_'' to match any word);
\item The head word (on the first example, the POS-tag must match the ``VER.*'' Java-compliant regular expression)
\item The dependency label
\end{enumerate}

\subsection{Behavior}

{\bf Warning:} the order you write the ``triplets'' is important:
in fact, the system will search for {\bf every} occurrences of the first triplet,
and for each such occurrence, it will search for {\bf the first} occurrence of the second,
third, ... triplets.
This behavior can be changed to searching for every occurrences of the second, third, ...
triplets by checking the first check-box on the query-editing window.

\subsection{Special constraints}

\subsubsection{Match the closest word: dep1..closest..dep0}

This matches only one possible triplet T1: the one with the closest dep1 from dep0.
This typically is used in conjunction with another left or right constraint: e.g., dep1<dep0
so that only the closest on the left is matched. Without any left nor right constraint,
the closest dep1 will match, and in case of left-right conflict about the closest, one of them
is chosen, and it is undefined which one.

This constraint breaks the recursive search and doubles the search cost, so it shall be used
with care. In particular:
\begin{itemize}
\item There must be AT MOST ONLY ONE closest constraint per query; otherwise, the search might crash.
\item The constraint MUST have the form depX..closest..depY with X>Y !! Otherwise, the search may crash.
\end{itemize}

\end{document}
